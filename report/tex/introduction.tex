\section{Action Market}
\noindent Our research considers an action market made up of a group of agents trying to achieve a common goal of maximizing their social welfare. Agents in this market act as both actors and observers. As actors, agents will perform an action and as a result, they receive reward from the environment and feedback from observers. As observers, agents observe the actions of an actor and will provide positive or negative feedback. \\


\noindent An example of such an action market is where a group of agents are collectively trying to pick up as many apples as possible within a fixed time frame in an orchard (i.e. a grid) where apples spawn with some underlying probabilistic model. Actors will perform an action (e.g. moving up, down, left, right, or staying in same square) by considering its own experiences and previous feedback received from observers. On the other hand, observers provide (positive or negative) feedback to influence the actions taken by actors in the future in order to maximize the number of apples picked, the social welfare in this example. 

\subsection{Modelling of Action Market}

The objectives of the agents in the action market are as follows. \\

\noindent Agents acting as observers aim to influence the future actions of the actors in a manner that maximizes social welfare. They achieve this by providing positive or negative feedback based on the action performed by an actor. This implies that feedback is delivered after the action has been performed. This feedback helps actors learn the value of the actions performed for observers. With this said, observers have limited communication capacity to be allocated among all actors. For an observer, the communication capacity imposes restrictions on the rate of delivering feedback to each actor. \\ 

\noindent Agents acting as actors aim to act in a way that maximizes their individual reward while satisfying observers' expectations. While actors try to maximize their individual reward, they also try to garner positive feedback provided by observers. They achieve this by learning observers' expectations through previous feedback received and by referring to these expectations when performing an action in the future. \\

\noindent In summary, the modelling assumptions that we make for the actor market are as follows.
\begin{enumerate}
    \item Objective of observers: Observers aim to maximize social welfare by influencing other actors by providing feedback to actions performed by actors.
    \item Objective of actors: Actors aim to maximize individual reward while satisfying observers' expectations.
    \item Order of action and feedback: A feedback is provided by an observer to an actor after an action is performed by the actor.
    \item Communication capacity as a limited resource: Observers have limited communication capacity to be allocated to different actors. This limits the rate at which feedback is delivered.
\end{enumerate}

\subsection{Terminology: Actors and Observers}
As mentioned above, agents behave both as actors and observers,. However, when we focus on an agent performing an action, we refer to that agent as an actor. Likewise, when we focus on an agent providing a feedback, we refer to that agent as an observer. By using these terminologies, we can distinguish between the two activities performed by agents, namely performing an action and providing feedback for another agent's action. \\ 


% \section{Mathematical Model}

% 